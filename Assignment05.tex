\documentclass[tikz,border=2pt,png]{article}
\usepackage{tikz}
\usepackage{amsmath}
\usepackage{amssymb}
\usepackage{pgfplots}
\begin{document}
\title{ASSIGNMENT 5}
\author{Showqeen Yousuf}
\date{\today}
\maketitle

\begin{itemize}
\item{\textbf{Exercise 2.55:}}\\

Let A and B be the centres of two circles of equal Radii 3 such that each one of them passes through the centre of the other. Let them intersect at C and D. Is AB \perp CD:}\\


\item{\textbf{Solution:}}\\

$ 

Given, Radii of two circles=3cm$\\

Now,\\

\textit{Steps of Construction are:}\\
$

1: Taking a fixed point A as radius=3cm.\\

2: Draw a circle O.\\

3: Again taking any point B on circumference of circle C, with radius=3cm draw another circle,intersecting circle O at points C and D.\\

4: Join AB  and CD.\\
$


\begin{centre}
 \begin{tikzpicture}
 \draw(-7,0)--(7,0) node[anchor=north west]{x axis};
 \draw(0,-5)--(0,5) node[anchor=south east]{y axis};
 \node at (-0.2,0.2){$A$};
 \node at (3.2,0.2){$B$};
 \draw[blue] (0,0)--(3,0); 
 \draw (1.5,2.6)--(1.5,-2.6);
 \node at (1.5,2.8){$C$};
 \node at (1.5,-2.8){$D$};
 \draw (0,0) circle (3 cm);  
 \draw (3,0) circle (3 cm);
 \end{tikzpicture}
\end{centre}\\

\terefore{}& \hspace{3 cm}By Symmetry AB is perpendicular to CD(AB \perp CD).

\newpage
\item{\textbf{Question 2.56:}}\\

Construct a tangent to a circle of radius 4 units from a point on concentric circle of radius 6 units:

\item{\textbf{Solution:}}\\

Given the radii of two concentric circles of radius 4 units and 6 units respectively,\\

\textit{Steps of Construction:}\\
$

1: Draw a circle of radius 4 units with centre O.\\

2: Draw a circle of radius 6 units wit taking O as its centre. Locate a point P on this circle and join OP.\\

3: Bisect OP. Let M be the midpoint of PO.\\

4: Taking M as its centre and MO as its radius, draw a cricle. Let it intersect the given circle at points Q and R.\\

5: Join PQ and PR. PQ and PR are the required tangents.
$\\
 
   From the construction required tangents are given below:\\
   
   \begin{centre}
 \begin{tikzpicture}
 \draw(-7,0)--(7,0) node[anchor=north west]{x axis};
 \draw(0,-7)--(0,7) node[anchor=south east]{y axis};
 \node at (0.3,0.3){$O$};
 \node at (6.2,0.2){$P$};
 \node at (3.2,0.2){$M$};
 \draw (0,0)--(6,0);
 \draw[dashed](3,0) circle (3 cm);
 \draw[dashed](3,3.8)--(3,-3.8);
 \node at (2.7,3.2){$Q$};
 \node at (2.7,-3.2){$R$};
 \draw[red] (2.7,2.98)--(6,0);
 \draw[red] (2.7,-2.98)--(6,0);
 \draw(0,0) circle (4 cm);  
 \draw (0,0) circle (6 cm);
 \end{tikzpicture}
\end{centre}\\
 
\textit{ PQ and PR are the two tangents to the internal circle from a point P located on the circumference of outer circle.}

\end{itemize}
\end{document}